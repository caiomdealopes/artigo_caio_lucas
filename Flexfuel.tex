\documentclass[
	% -- opções da classe memoir --
	article,			% estilo de artigo
	12pt,				% tamanho da fonte
		% capítulos começam em pág ímpar (insere página vazia caso preciso)
	oneside,			% para impressão em verso e anverso. Oposto a twoside
	a4paper,			% tamanho do papel. 
	% -- opções da classe abntex2 --
	%chapter=TITLE,		% títulos de capítulos convertidos em letras maiúsculas
	%section=TITLE,		% títulos de seções convertidos em letras maiúsculas
	%subsection=TITLE,	% títulos de subseções convertidos em letras maiúsculas
	%subsubsection=TITLE,% títulos de subsubseções convertidos em letras maiúsculas
	% -- opções do pacote babel --
	english,			% idioma adicional para hifenização
	%french,				% idioma adicional para hifenização
	%spanish,			% idioma adicional para hifenização
	brazil				% o último idioma é o principal do 
	]{abntex2}

% ---
% PACOTES
% ---

% ---
% Pacotes fundamentais 
% ---
\usepackage{cmap}				% Mapear caracteres especiais no PDF
\usepackage{lmodern}			% Usa a fonte Latin Modern
\usepackage[T1]{fontenc}		% Selecao de codigos de fonte.
\usepackage[utf8]{inputenc}		% Codificacao do documento (conversão automática dos acentos)
\usepackage{indentfirst}		% Indenta o primeiro parágrafo de cada seção.
\usepackage{color}				% Controle das cores
\usepackage{graphicx}			% Inclusão de gráficos
\usepackage{multirow}			% Tabelas
\usepackage{nicefrac}			% Frações
% ---
\usepackage{enumerate}
\usepackage{array}
\usepackage{amsmath}
\usepackage{amsfonts}
\usepackage{amsthm}
\usepackage{scalefnt}
\usepackage{longtable}
% ---
% Pacotes adicionais, usados apenas no âmbito do Modelo Canônico do abnteX2
% ---
\usepackage{lipsum}				% para geração de dummy text
%\usepackage[english,brazil]{babel}
% ---
% ---
% Pacotes de citações
% ---
\usepackage[alf,bibjustif]{abntex2cite}	% Citações padrão ABNT (bibjustif) Forçar alinhamento à direita
% Page length commands go here in the preamble


\hypersetup{hidelinks}
\renewcommand{\baselinestretch}{1.0} % 1.5 denotes double spacing. Changing it will change the spacing


\date{}
\begin{document}
\title{Análise da elasticidade da demanda por combustíveis frente à entrada da tecnologia \textit{Flex Fuel} }
\author{Caio Matteucci de Andrade Lopes\thanks{caiolopes@ufpr.br} 
    \and Lucas Lourenço Lopes\thanks{lucasllopes@ufpr.br } }
   

\maketitle

\begin{abstract}
O objetivo deste trabalho é analisar a demanda por combustíveis no município de São Paulo. Para isso, estimaremos a elasticidade da demanda da gasolina e do etanol com base nos dados fornecidos pela Agência Nacional de Petróleo (ANP). A análise econométrica realizada será o Método dos Momentos generalizados (GMM), Vetor de Correção de Erros com Threshold (TVECM).      \\ \\
\textbf {Palavras-chave}: Elasticidade, Consumo Gasolina, Flex Fuel, Álcool
\end{abstract}

%{
\selectlanguage{english}
\begin{abstract}
The objective of this work is to analyze the fuel demand in the city of São Paulo. To do so, we will estimate the elasticity of demand for gasoline and ethanol based on data provided by the National Petroleum Agency (ANP). The econometric analysis will be the Generalized Moments Method (GMM) and Threshold Error Correction Vector (TVECM). \\ \\
\textbf {Keywords}: Elasticity, Fuel Consumption, Flex Fuel, Alcohol
\end{abstract}
\selectlanguage{brazil}



% ----------------------------------------------------------
% Introdução
% ----------------------------------------------------------

% ----------------------------------------------------------
% Introdução
% ----------------------------------------------------------
\section{Introdução}



% ----------------------------------------------------------
% Capitulo de textual  
% ----------------------------------------------------------
\section{Revisão Bibliográfica}

% ----------------------------------------------------------
% Capitulo de textual  
% ----------------------------------------------------------
\section{Objetivos e Motivação}


% ----------------------------------------------------------
% Capitulo de textual  
% ----------------------------------------------------------
\section{Dados}

% ----------------------------------------------------------
% Capitulo de textual  
% ----------------------------------------------------------
\section{Metodologia}
\subsection{Abordagem teórica}

 

% ----------------------------------------------------------
% Capitulo de textual  
% ----------------------------------------------------------
\subsection{Modelo Econométrico}


% ----------------------------------------------------------
% Capitulo de textual  
% ----------------------------------------------------------
\section{Resultados}


% ----------------------------------------------------------
% Capitulo de textual  
% ----------------------------------------------------------
\section{Conclusão}

\cleardoublepage
% ----------------------------------------------------------
% ELEMENTOS PÓS-TEXTUAIS
% ----------------------------------------------------------
\postextual

% ----------------------------------------------------------
% Referências bibliográficas
% ----------------------------------------------------------

\bibliography{refcaio}
\cleardoublepage

\appendix

\chapter{Apêndice}

\label{LastPage}
\end{document}
